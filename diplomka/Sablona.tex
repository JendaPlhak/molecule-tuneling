
%%%%%%%%%%%%%%%%%%%%%%%%%%%%%%%%%%%%%%%%%%%%%%%%%%%%%%%%%%%%%%%%%%%%%%%%%%%%%%%%%%%%%%%%%%%%%%%%%%%%%%%
% Sablona Bc+Mgr+RNDr (CZ) pro PriF MU                                                              %%%
% Autor: Petr Zemanek (zemanekp@math.muni.cz)                                                       %%%
% Pripominky, dotazy, namety smerujte na diskusni forum: https://is.muni.cz/auth/df/sablona-prif    %%%
% Typeset in LaTeX-2e                                                                               %%%
% Verze: 1.9 (4. brezna 2016)                                                                       %%%
%%%%%%%%%%%%%%%%%%%%%%%%%%%%%%%%%%%%%%%%%%%%%%%%%%%%%%%%%%%%%%%%%%%%%%%%%%%%%%%%%%%%%%%%%%%%%%%%%%%%%%%

% \documentclass[11pt,a4paper,oneside,final]{book} %% PRO JEDNOSTRANNY TISK
% \documentclass[11pt,a4paper,twoside,final]{book} %% PRO OBOUSTRANNY TISK
% \documentclass[12pt,a4paper,oneside,final]{book} %% PRO JEDNOSTRANNY TISK
\documentclass[12pt,a4paper,twoside,final]{book} %% PRO OBOUSTRANNY TISK

%%%%%%%%%%%%%%%%%%%%%%%%%%%%%%%%%%%%%%%%%%%%%%%%%%%%%%%%%%%%%%%%%%%%%%%%%%%%%%%%%%%%%%%%%%%%%%%%%%%%%%
%%%%%%%%%%%%%%%%%%%%%%%%%%%%%%%%%%%%%% ZAKLADNI NASTAVENI %%%%%%%%%%%%%%%%%%%%%%%%%%%%%%%%%%%%%%%%%%%%

%%%%%%%%%%%%
%% Zvolte kodovani dokumentu
%%%%%%%%%%%%

\usepackage[utf8]{inputenc}
% \usepackage[cp1250]{inputenc} %% NASTAVENI PRO WINDOWS
% \usepackage[T1]{fontenc} %% puvodni evropske fonty->spatne umistene hacky nad c, d s hackem ma za sebou mezeru
\usepackage[IL2]{fontenc} %fonty vyladene pro cestinu/slovenstinu

%%%%%%%%%%%%
%% Nastavte si jazyk dokumentu
%% lze pouzit volbu babel (pro pdflatex a latex) nebo bez babelu (pro pdfcslatex a cslatex
%%
%% pri volbe bez babelu bude zlobit v ukazkovych souborech prikaz ``\shorthandoff{-}'' a ``\shorthandoff{-}''
%% ty staci vymazat a je to bez problemu
%%
%% pri volbe ``\usepackage[english]{babel}'' nefunguji ceske uvozovky pomoci \uv{..} - staci nahradit ``..''
%%%%%%%%%%%%

\usepackage[czech]{babel}
% \usepackage[slovak]{babel}
% \usepackage[english]{babel}
% \usepackage{czech}
% \usepackage{slovak}

%%%%%%%%%%%%
%% Muzete si zkusit zmenit font dokumentu - pak ale musite v souboru zemnit nektera nastaveni (nebot je nastaveno
%% \overfullrule10pt je preteceni textu ihned vyznaceno cernym obdelnikem)
%% Veskere nastaveni je testovano pro pouzitou volbu
%%%%%%%%%%%%

% \usepackage{bookman}
% \usepackage{charter}
% \usepackage{fourier}
% \usepackage{mathpazo}
% \usepackage{newcent}
% \usepackage{palatino}
% \usepackage{utopia}
\usepackage{mathptmx}  %% volne dostupny font Adobe Times Roman
%  \usepackage[mtbold,mtplusscr,mtpluscal]{mathtime} % komercni matematicky font, dostupne pouze na UMS


%%%%%%%%%%%%%%%%%%%%%%%%%%%%%%%%%%%%%%%%%%%%%%%%%%%%%%%%%%%%%%%%%%%%%%%%%%%%%%%%%%%%%%%%%%%%%%%%%%%%
%%%%%%%%%%%%%%%%%%%%%%%%%%%%%%%% BALIKY POTREBNE PRO SABLONU %%%%%%%%%%%%%%%%%%%%%%%%%%%%%%%%%%%%%%%

\usepackage{longtable,lipsum}
%%% balik ``longtable`` je potrebny pro sazeni kapitoly s pouzitym znacenim, jinak neni potreba
%%% balik ``lipsum`` je pouze pro sablonu, aby se generoval nahodny text - pro samotnou praci je mozne jej odstranit

%%%%%%%%%%%%%%%%%%%%%%%%%%%%%%%%%%%%%%%%%%%%%%%%%%%%%%%%%%%%%%%%%%%%%%%%%%%%%%%%%%%%%%%%%%%%%%%%%%%%
%%%%%%%%%%%%%%%%%%%%%%%%%%%%% BALIKY POTREBNE PRO MATEMATIKU %%%%%%%%%%%%%%%%%%%%%%%%%%%%%%%%%%%%%%%

\usepackage{amsmath,amssymb,amsthm}
\usepackage{amsmath}
\usepackage{enumitem}
\usepackage{dirtytalk}
\usepackage{lastpage}
\usepackage{siunitx}
\usepackage{algorithmicx}
\usepackage{algorithm}
\usepackage{relsize}
\usepackage[noend]{algpseudocode}
\usepackage{xcolor}


%%%%%%%%%%%%%%%%%%%%%%%%%%%%%%%%%%%%%%%%%%%%%%%%%%%%%%%%%%%%%%%%%%%%%%%%%%%%%%%%%%%%%%%%%%%%%%%%%
%%%%%%%%%%%%%%%%%% NASTAVENI POVINNYCH CASTI DLE SMERNICE DEKANA %%%%%%%%%%%%%%%%%%%%%%%%%%%%%%%%

%%%%%%%%%%%%
%% NASTAVTE KONKRETNI ADAJE
%%%%%%%%%%%%

\usepackage[Mgr,Barevne]{sci.muni.thesis}
%% Mozne volby:
%% Bc - pro bakalarskou praci
%% Mgr - pro diplomovou praci
%% RNDr - pro rigorozni praci
%% Barevne - pro dokument s barevnymi odkazy
%% Tisk - pro dokument v cernobile barve

\NazevUstavu{Ústav matematiky a statistiky}{Department of Mathematics and Statistics}

\RokOdevzdaniPrace{2017}

\AkademickyRok{2016/2017}

\Autor{Jan Plhák}{Bc. Jan Plhák}

\NazevPrace{
    Automatický doking ligandu do tunelu v proteinu } {
    Automatický doking ligandu do tunelu v proteinu } {
    Automatic docking of a ligand into a tunnel in a protein
}
%% Pokud nazev prace obsahuje nejake matematicke formule, tak je kvuli velikosti fontu pouzit prikaz \scalebox a nazev
%% zalamat rucne - toto plati pouze pro nazev prace na titulni list
%% {Rozbor řešení rovnice \scalebox{1.15}{${y''=f(x)\,x^{\scalebox{0.4}2}}$} s~počátečními podmínkami
%% \scalebox{1.15}{$y(-\infty)=0$} a~\scalebox{1.15}{$y'(\infty)=0$}}
%%
%% pro psani druhe polozky, tj. {Název práce} je nutne mit na pameti, ze matematicke symboly nelze do zalozek v PDF
%% vlozit - proto slouzi prikaz \texorpdfstring{toto se vysazi}{toto se vlozi do zalozek v PDF}
%% nektere symboly vlozit lze, viz kapitolu 50 v dokumentaci
%% http://mirrors.ctan.org/macros/latex/contrib/hyperref/hyperref.pdf
%% viz take http://orgmode.org/worg/org-symbols.html
%%

\VedouciPraceSTituly{prof. RNDr. Luděk Matyska, CSc.} %% Neni potreba pro rigorozni prace

\StudijniProgram{Matematika}{Mathematics}

\StudijniObor{Matematika s informatikou}{Mathematics with Informatics}

\PocetStran{??\,$+$\,??}

\KlicovaSlova{Klíčové slovo; Klíčové slovo; Klíčové slovo; Klíčové slovo; Klíčové slovo; Klíčové slovo;
Klíčové slovo; Klíčové slovo}{Keyword; Keyword; Keyword; Keyword; Keyword; Keyword; Keyword;
Keyword; Keyword}

\Abstrakty%
{V této bakalářské/diplomové/rigorózní práci se věnujeme ...}%
{In this thesis we study ...}

\TextPodekovani%
{Na tomto místě bych chtěl(-a) poděkovat ...}

\TextProhlaseni%
{Prohlašuji, že jsem svoji diplomovou práci vypracoval samostatně s~využitím informačních
zdrojů, které jsou v práci citovány.}

\DatumProhlaseni{15. května 2017}

%%%%%%%%%%%%
%% Konkrétní příklad
% \NazevUstavu{Ústav matematiky a statistiky}{Department of Mathematics and Statistics}
%
% \RokOdevzdaniPrace{2012}
%
% \AkademickyRok{2011/12}
%
% \Autor{Petr Zemánek}{Mgr. Petr Zemánek, Ph.D.}
%
% \NazevPrace
% {Rozbor řešení rovnice \scalebox{1.15}{${y''=f(x)\,x^{\scalebox{0.4}2}}$} s~počátečními podmínkami
% \scalebox{1.15}{$y(-\infty)=0$} a~\scalebox{1.15}{$y'(\infty)=0$}}
% {Rozbor řešení rovnice \texorpdfstring{$y''=f(x)\,x^2$}{y''=f(x)x\texttwosuperior} s~počátečními podmínkami
% \texorpdfstring{$y(-\infty)=0$}{y(-∞)} a~\texorpdfstring{$y'(\infty)=0$}{y'(∞)=0}}
% {Analysis of solution of the equation $y''=f(x)\,x^2$ with the initial conditions $y(-\infty)=0$ and $y'(\infty)=0$}
%
% {Lineární diferenciální rovnice {\it n}-tého řádu s~konstantními koeficienty a~jejich aplikace}{Linear
% Differential Equations of {\it n}-th Order with Constant Coefficients and Their Applications}
%
% \VedouciPraceSTituly{Mgr. Petr Zemánek, Ph.D.}
%
% \StudijniProgram{Matematika}{Mathematics}
%
% \StudijniObor{Matematika}{Mathematics}
%
% \PocetStran{xiii\,$+$\,14}  %%% tyto hodnoty odpovidaji teto sablone
%
% \KlicovaSlova{Klíčové slovo; Klíčové slovo; Klíčové slovo; Klíčové slovo; Klíčové slovo; Klíčové slovo; Klíčové
% slovo; Klíčové slovo}{Keyword; Keyword; Keyword; Keyword; Keyword; Keyword; Keyword; Keyword; Keyword}
%
% \Abstrakty%
% {V této bakalářské/diplomové práci se věnujeme ...}%
% {In this thesis we study ...}
%
% \TextPodekovani%
% {Na tomto místě bych chtěl(-a) poděkovat ...}
%
% \TextProhlaseni{
% Prohlašuji, že jsem svoji bakalářskou/diplomovou/rigorózní práci vypracoval(-a) samostatně s~využitím informačních
% zdrojů, které jsou v práci citovány.}
%
% \DatumProhlaseni{27. ledna 2012}
%

%%%%%%%%%%%%%%%%%%%%%%%%%%%%%%%%%%%%%%%%%%%%%%%%%%%%%%%%%%%%%%%%%%%%%%%%%%%%%%%%%%%%%%%%%%%%%%%%%%%%%%
%%%%%%%%%%%%%%%%%%%%%%%%%%%%%%%%%%%% VLASTNI PRIKLAZY %%%%%%%%%%%%%%%%%%%%%%%%%%%%%%%%%%%%%%%%%%%%%%%%
%%%%%%%%%%%%%% ZDE SI MUZETE DEFINOVAT VLASTNI PRIKAZY PRO SNAZSI SAZENI CELEHO TEXTU %%%%%%%%%%%%%%%%

\newcommand{\Cbb}{\mathbb{C}}
\newcommand{\Rbb}{\mathbb{R}}
\newcommand{\Zbb}{\mathbb{Z}}
\newcommand{\Nbb}{\mathbb{N}}

\newcommand{\angstrom}{\textup{\AA}}
\newcommand\todo[1]{\textcolor{red}{#1}}

\newcommand{\Tau}{\mathrm{T}}
\newcommand{\norm}[1]{\left\lVert#1\right\rVert}
\DeclareMathOperator{\dst}{dst}
\DeclareMathOperator{\dis}{distance}

% Pseudocode stuff
\renewcommand{\thealgorithm}{\arabic{chapter}.\arabic{algorithm}}
\newcommand{\Break}{\State \textbf{break} }

%%%%%%%%%%%%%%%%%%%%%%%%%%%%%%%%%%%%%%%%%%%%%%%%%%%%%%%%%%%%%%%%%%%%%%%%%%%%%%%%%%%%%%%%%%%%%%%%%%%%%%
%%%%%%%%%%%%%%%%%%%%%%%% VYTVOR POMOCNY SOUBOR PRO REJSTRIK %%%%%%%%%%%%%%%%%%%%%%%%%%%%%%%%%%%%%%%%%%

\makeindex

%%%%%%%%%%%%%%%%%%%%%%%%%%%%%%%%%%%%%%%%%%%%%%%%%%%%%%%%%%%%%%%%%%%%%%%%%%%%%%%%%%%%%%%%%%%%%%%%%%%%%%
%%%%%%%%%%%%%%%%%%%%%%% ZAPNUTI OPAKOVANI MATEMATICKYCH SYMBOLU %%%%%%%%%%%%%%%%%%%%%%%%%%%%%%%%%%%%%%

% %Podle Tesaříkova czech.ldf
\makeatletter
\def\Deleni{%
 \ifx\protect\@typeset@protect
   \ifhmode
     \ifinner
       \bbl@afterelse\bbl@afterelse\bbl@afterelse\cs@hyphen
     \else
       \bbl@afterfi\bbl@afterelse\bbl@afterelse\cs@firsthyphen
     \fi
   \else
     \bbl@afterfi\bbl@afterelse\cs@hyphen
   \fi
 \else
   \bbl@afterfi\cs@hyphen
 \fi }
\makeatother

%Opakování symbolů binárních operací a relací při zalomení řádku
%Autor: Josef Tkadlec tkadlec@fel.cvut.cz

\relpenalty     =10000      % aby se nelámalo v jiných než ošetřených
\binoppenalty   =10000
\exhyphenpenalty=1000       % aby spíše nouzově (implicitně je 50)
                           % "lokálně" lze zakázat {...}

\def\neq {\mathrel{\not=}}  % aby nedocházelo k lámání \not=/=
\let\ne=\neq

\def\OpakujPrikaz #1#2{\let #2=#1
 \def #1{#2\nobreak\discretionary{}{\hbox{$#2$}}{}}}
\def\OpakujZnak #1#2{\mathchardef #2=\mathcode`#1
 \activedef #1{#2\nobreak\discretionary{}{\hbox{$#2$}}{}}
 \uccode`\~=0 \mathcode`#1="8000 }
%Doplnil Kuben pro nový czech.ldf \expandafter možná nemusí být
\def\OpakujZnakMinus #1#2{\mathchardef #2=\mathcode`#1
 \activedef #1{\ifmmode#2\nobreak\discretionary{}{\hbox{$#2$}}{}\else\expandafter\Deleni\fi }
 \uccode`\~=0 \mathcode`#1="8000 }
\def\activedef #1{\uccode`\~=`#1 \uppercase{\def~}}

\OpakujPrikaz {\neq }{\neqORI}  \let \ne=\neq
\OpakujPrikaz {\leq }{\leqORI}  \let \le=\leq
\OpakujPrikaz {\geq }{\geqORI}  \let \ge=\geq
\OpakujPrikaz {\cup }{\cupORI}
\OpakujPrikaz {\cap }{\capORI}
\OpakujPrikaz {\times }{\timesORI}
\OpakujPrikaz {\subset }{\subsetORI}
\OpakujPrikaz {\subseteq }{\subseteqORI}
\OpakujPrikaz {\supset }{\supsetORI}
\OpakujPrikaz {\supseteq }{\supseteqORI}

\OpakujPrikaz {\cdot }{\cdotORI}
\OpakujPrikaz {\setminus }{\setminusORI}

\OpakujZnak <{\lessORI}
\OpakujZnak >{\greaterORI}
\OpakujZnak +{\plusORI}
\AtBeginDocument {\OpakujZnak ={\eqORI} \OpakujZnakMinus -{\minusORI}}

%%%%%%%%%%%%%%%%%%%%%%%%%%%%%%%%%%%%%%%%%%%%%%%%%%%%%%%%%%%%%%%%%%%%%%%%%%%%%%%%%%%%%%%%%%%%%%%%%%%%%%
%%%%%%%%%%%%%%%%%%%%%%%%%%%%%%%%%% ZACATEK DOKUMETU %%%%%%%%%%%%%%%%%%%%%%%%%%%%%%%%%%%%%%%%%%%%%%%%%%
%%%%%%%%%%%%%%%%%%%%%%%%%%%%%%%%%%%%%%%%%%%%%%%%%%%%%%%%%%%%%%%%%%%%%%%%%%%%%%%%%%%%%%%%%%%%%%%%%%%%%%

\begin{document}

%%%%%%%%%%%%%%%%%%%%%%%%%%%%%%%%%%%%%%%%%%%%
%%%%%%%%%%%% POVINNE CASTI %%%%%%%%%%%%%%%%%

\VytvorPovinneStrany

%%%%%%%%%%%%%%%%%%%%%%%%%%%%%%%%%%%%%%%%%%%%%%%%%%%%%%%%%%%%%%%%%%%%%%
%%%%%%%%%%%% POVINNE CASTI PRO PRACE PSANE SLOVENSKY %%%%%%%%%%%%%%%%%
%
% \NazevPraceSLOVENSKY{...}
%
% \NazevUstavuSLOVENSKY{...}
%
% \StudijniProgramSLOVENSKY{...}
%
% \StudijniOborSLOVENSKY{***}
%
% \KlicovaSlovaSLOVENSKY{bbb}
%
% \AbstraktSLOVENSKY{...}
%
% \VytvorPovinneStranySLOVENSKY
%
%%%%%%%%%%%%%%%%%%%%%%%%%%%%%%%%%%%%%%%%%%%%%%%%%%%%%%%%%%%%%%%%%%%%%%
%%%%%%%%%%%% POVINNE CASTI PRO RIGOROZNI PRACE %%%%%%%%%%%%%%%%%%%%%%%

% \VytvorPovinneStranyRigorozniPrace

%%%%%%%%%%%%%%%%%%%%%%%%%%%%%%%%%%%%%%%%%%%%%%%%%%%%%%%%%%%%%%%%%%%%%%%%%%%%%%%%%%%%%
%%%%%%%%%%%% POVINNE CASTI PRO RIGOROZNI PRACE PSANE SLOVENSKY %%%%%%%%%%%%%%%%%%%%%%%

% \VytvorPovinneStranyRigorozniPraceSVK

%%%%%%%%%%%%%%%%%%%%%%%%%%%%%%%%%%%%%%%%%%%%%%%%%%%%%%%%%%%%%%%%%%%%%%%%%%%%%%%%%%%%%%%%%%%%%%%%%%%
%%%%%%%%%%%%%%%%%%%%%%%%%%%%%%%%%% ABSTRAKT + ABSTRACT %%%%%%%%%%%%%%%%%%%%%%%%%%%%%%%%%%%%%%%%%%%%

\AbstraktyNaJedneStrane

%%%%%%%%%%%%%%%%%%%%%% POKUD SE ABSTRAKTY NEVEJDOU NA JEDNU STRANU, POUZIJTE TOTO NASTAVENI %%%%%%%

% \AbstraktyNaDvouStranach

%%%%%%%%%%%%%%%%%%%%%%%%%%%%%%%%%%%%%%%%%%%%%%%%%%%%%%%%%%%%%%%%%%%%%%%%%%%%%%%%%%%%%%%%%%%%%%%%%%%
%%%%%%%%%%%%%%%%%%%%%%%%%%%%%% ABSTRAKT + ABSTRAKT + ABSTRACT %%%%%%%%%%%%%%%%%%%%%%%%%%%%%%%%%%%%%

% \AbstraktyNaJedneStraneSLOVENSKY

%%%%%%%%%%%%%%%%%%%%%% POKUD SE ABSTRAKTY NEVEJDOU NA JEDNU STRANU, POUZIJTE TOTO NASTAVENI %%%%%%%

% \AbstraktyNaViceStranachSLOVENSKY


%%%%%%%%%%%%%%%%%%%%%%%%%%%%%%%%%%%%%%%%%%%%%%%%%%%%%%%%%%%%%%%%%%%%%%%%%%%%%%%%%%%%%%%%%%%%%%%%%%%
%%%%%%%%%%%%%%%%%%%%%%%%%%%%%%%%%%%%%% SEM VLOZIT ZADANI %%%%%%%%%%%%%%%%%%%%%%%%%%%%%%%%%%%%%%%%%%
%%% OSKENUJTE JEJ A VE FORMATU PDF VLOZTE DO STEJNE SLOZKY (SOUBOR POJMENUJTE NAPR. zadani.pdf) %%%
%%%%%% PAK ''ZAPROCENTUJTE'' RADEK S \SemVlozitZadani A ODPROCENTUJTE RADEK S \VlozZadani %%%%%%%%%
%%%%%%%%%%%%%%%%%%%%%%%%%%%%%%%%%%%%%%%%%%%%%%%%%%%%%%%%%%%%%%%%%%%%%%%%%%%%%%%%%%%%%%%%%%%%%%%%%%%

\SemVlozitZadani

% \VlozZadani{NazevSouboruSeZadanim}
% \VlozZadani{zadani.pdf}

%%%%%%%%%%%%%%%%%%%%%%%%%%%%%%%%%%%%%%%%%%%%%%%%%%
%%%%%%%%%% PODEKOVANI + PROHLASENI %%%%%%%%%%%%%%%

\PodekovaniAProhlaseni


%%%%%%%%%%%%%%%%%%%%%%%%%%%%%%%%%%%%%%%%%%%%%%%%%%%%%%
%%%%%%%%%% PODEKOVANI + PROHLASENI SVK %%%%%%%%%%%%%%%

% \PodekovaniAProhlaseniSLOVENSKY

%%%%%%%%%%%%%%%%%%%%%%%%%%%%%%%%%%%%%%%%%%%%%%%%%%
%%%%%%%%%%%%% POUZE PROHLASENI %%%%%%%%%%%%%%%%%%%
%
% \ProhlaseniBezPodekovani
%
%%%%%%%%%%%%%%%%%%%%%%%%%%%%%%%%%%%%%%%%%%%%%%%%%%
%%%%%%%%%%%%%%%%%%% OBSAH %%%%%%%%%%%%%%%%%%%%%%%%

\pdfbookmark{Obsah}{Obsah}
\VytvorObsah
\cleardoublepage

%%%%%%%%%%%%%%%%%%%%%%%%%%%%%%%%%%%%%%%%%%%%%%%%%%%%%%%%%%%%%%%%%%%%%%%%%%%%%%%%%%%%%%%%%%%%%%%%%%%
%%%%%%%%%%%%%%%%%%%%%%%%%%%%%%%%%%% TEXT PRACE %%%%%%%%%%%%%%%%%%%%%%%%%%%%%%%%%%%%%%%%%%%%%%%%%%%%

\theoremstyle{plain}
\newtheorem{vet}{Věta}[chapter] % reset theorem numbering for each chapter
\newtheorem{lem}[vet]{Lemma} % definition numbers are dependent on theorem numbers
\newtheorem{dus}[vet]{Důsledek} % definition numbers are dependent on theorem numbers

\theoremstyle{definition}
\newtheorem{defi}[vet]{Definice} % definition numbers are dependent on theorem numbers
\newtheorem{pozn}[vet]{Poznámka} % definition numbers are dependent on theorem numbers
\newtheorem{pri}[vet]{Příklad} % same for example numbers

\makeatletter
\@addtoreset{algorithm}{chapter}% algorithm counter resets every chapter
\makeatother

\HlavickaUvod
\pdfbookmark{Úvod}{Uvod}
\vloz{text_prace/01_Uvod}
\cleardoublepage

% \HlavickaZnaceni
% \pdfbookmark{Přehled použitého značení}{Prehled pouziteho znaceni}
% \vloz{text_prace/02_Znaceni}
% \cleardoublepage

\renewcommand{\chaptermark}[1]{\markboth{\thechapter. #1}{}}
\renewcommand{\sectionmark}[1]{\markright{\thesection. #1}{}}
\HlavickaKapitoly
\vloz{text_prace/03_Kapitola_01}
\cleardoublepage

\HlavickaKapitoly
\vloz{text_prace/04_Kapitola_02}
\cleardoublepage

% \HlavickaZaver
% \pdfbookmark{Závěr}{Zaver}
% \vloz{text_prace/05_Zaver}
% \cleardoublepage

\HlavickaPriloha
\pdfbookmark{Appendix}{Appendix}
\vloz{text_prace/06_Priloha}
\cleardoublepage

\renewcommand{\bibname}{Seznam použité literatury}
\HlavickaLiteratura
\vloz{text_prace/07_Literatura}
\cleardoublepage

% \renewcommand{\indexname}{Rejstřík}
% \HlavickaRejstrik
% \VytvorRejstrik
%%
%% pro vytvoreni rejstriku se spravny ceskym razenim pouzijte
%% csindex -d -h -k -z il2 nazev_souboru.idx
%%
%% nebo
%% texindy -I latex -L czech -M lang/czech/utf8 nazev_souboru.idx
%%
%% na Ustavu matematiky a statistiky zadejte
%% /opt/texlive/2010/bin/i386-linux/texindy -I latex -L czech -M lang/czech/utf8 nazev_souboru.idx

%%%%%%%%%%%%%%%%%%%%%%%%%%%%%%%%%%%%%%%%%%%%%%%%
%%%%%%%%%%% PRAZDNA STRANA NA ZAVER %%%%%%%%%%%%
%%%%%%%%%%%%%%%%%%%%%%%%%%%%%%%%%%%%%%%%%%%%%%%%

\newpage
\thispagestyle{empty}
\fancyhf{}
\newpage
\mbox{}

\end{document}
