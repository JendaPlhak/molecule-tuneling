\chapter*{Úvod}
\addcontentsline{toc}{chapter}{Úvod}

Vědět jakým způsobem spolu budou interagovat různé dvojice protein-ligand je
velice důležité pro pochopení řady fundamentálních biochemických procesů, které
nalézají uplatnění v mnoha praktických odvětvích jako je například farmacie.
Ligand interaguje s proteinem v jeho \texttt{aktivním místě} (centru), kde dochází
k enzymové reakci. Příkladem může být inhibice aktivního místa proteinu, které
jinak umožňuje viru napadnout buňku. Simulace navázání ligandu (ligand vstoupí
do aktivního místa a vznikne stabilní vazba mezi ligandem a proteinem) a jeho
uvolnění (uvolnění ligandu ze stabilního komplexu) je užitečná v mnoha praktických
aplikacích, neboť nám umožňuje indetifikovat ligandy, které se budou navazovat
v daném proteinu nejrychleji, případně získané informace můžeme použít k úpravě
ligandu tak aby se navazoval rychleji nebo modifikovat samotný protein aby k
navázání docházelo snáz nebo hůř podle toho jaké chování je žádoucí. Aktivní
centrum proteinu je mnohdy ukryto hluboko uvnitř molekuly, takže aby se ligand
mohl navázat, musí nejprve projít tzv. tunelem, který vede z povrchu proteinu
k jeho aktivnímu centru. V takovém případě je potřeba analyzovat, zda je
pravděpodobné, aby se ligand do aktivního centra skze tunel dostal.

Chemické systémy, jakým je protein interagující s ligandem, se řídí druhým
termodynamickým zákonem a mají proto tendenci minimalizovat jejich
celkovou potenciální energii. Prakticky to znamená, že nejpravděpodobnější
\textit{konformace} molekul (tj. pozice jejich atomů) jsou ty, která mají minimální
potenciální energii. Molekulární systém se ale může nacházet také v nějakém
lokálním energetickém minimu a s určitou pravděpodobností mezi těmito lokálními
minimy může přecházet. Pravděpodobnost přechodu pak ovlivňuje teplota systému a
velikost energetické bariéry, která musí být při přechodu překonána, přičemž platí
že čím menší bariéra, tím vyšší pravděpodobnost přechodu. Tím pádem pokud známe
energetický profil průchodu ligandu tunelem proteinu, tak můžeme kvantifikovat
pravděpodobnost s jakou ligand tunelem projde. Funkci která pro zadanou konformaci
spočítá aproximaci její potenciální energie nazýváme \textit{silové pole}. Analýzou
potenciální energie, kterou získáme ze silového pole, pak můžeme vypočítat
s jakou pravděpodobností se bude daná konformace vyskytovat v reálném chemickém
systému. Díky tomu je možné porovnat dva chemické komplexy tvořené
dvojicemi protein-ligand (jakými jsou třeba dvě možné konformace ligandu
v aktivním centru jednoho proteinu) a rozhodnout, který komplex má vyšší
pravděpodobnost zformování v reálném systému.

Pokud chceme studovat navázání a uvolnění ligandu z aktivního centra, potřebujeme
být schopni vyhodnotit potenciální energii ligandu procházejícího tunelem
z povrchu proteinu do jeho aktivního centra a naopak. Ligand se v aktivním
jádře proteinu naváže pokud v tomto místě existuje silné energetické minimum
a v průběhu průchodu tunelem nenarazí na žádnou významnou energetickou bariéru
(gradient tunelu by měl být s menšími lokálními výkyvy stále klesající od
vstupu do tunelu až na jeho konec v aktivním místě). Pokud tunel obsahuje nějakou
silnou odpudivou bariéru, je pravděpodobné, že se ji ligandu nepodaří překonat
a tunelem neprojde. Poznamenejme, že energetický profil tunelu je pro každý
ligand unikátní a je tedy potřeba jej pro každou variaci ligandu vyhodnotit
znovu.

Navázání ligandu v aktivním jádře proteinu se obvykle počítá pomocí takzvaného
molekulárního dokingu. Algoritmus molekulárního dokingu prochází konformační
prostor molekulárního komplexu protein-ligand a snaží lokalizovat energetická
minima. Výstupem molekulárního dokingu je jeden nebo více protein-ligand
konformačních komplexů společně s jejich potenciální energií. Díky tomu
uživatel získá informaci o tom, který ligand se na protein váže s nižší potenciální
energií a může se na základě této informace pokoušet ligand nebo protein
dále upravovat. Problémem je, že algoritmus molekulárního dokingu vypočítá pouze
statické pozice ligandu a proteinu - nevygeneruje \textit{trajektorii} ligandu
(pohyb ligandu tunelem v čase), což znamená, že tento algoritmus sám o sobě
nestačí na to abychom mohli studovat přenos ligandu z povrchu proteinu
do jeho aktivního centra skrze jeho tunel.

V této práci popíšeme některé části nové metody pro výpočet potenciální energie
trajektorie ligandu, která umožňuje studovat proces navázání i uvolnění ligandu
včetně jeho cesty skrze tunel k aktivnímu centru. Naše metode je založena na
algirtmu molekulárního dokingu, který používá tak, že iterativně dokuje ligand
podél trajektorie tunelu a vyhodnocuje jeho potenciální energii. Náš doking
pracuje s hybridními silovými poli - jedná se o kombinaci chemického silového
pole, které se používá k výpočtu potenciální energie komplexu protein-ligand,
a omezujícího silového pole, které zmenšuje velikost prostoru potenciálních
konformací ligandu. Díky tomu je pozice ligandu v každém místě tunelu
omezena pouze na oblast definovaného prostoru. Skrze tunel zadaného proteinu
může existovat velmi mnoho cest, po kterých můžeme ligand provést. Prostor
všech potenciálních cest proto prohledáváme pomocí heuristického algoritmu
s backtrackingem. Naše metoda byla implementována uživatelsky přívětivém
nástroji CaverDock, který je navržen tak, aby byl schopen maximalizovat utilizaci
paralelních výpočetních architektur.

Tato práce se bude primárně soustředit na informaticko-matematickou stranu
daného problému: poskytneme obecný náhled na celý algoritmus a pak se budeme
soustředit zejména na dvě jeho části, za jejichž návrh a implementaci byl
zodpovědný autor této diplomové práce. Konkrétně se jedná o algoritmus pro
diskretizaci tunelu a algoritmus pro výpočet konvergence ligandu. Paralelně
s touto prací vzniká ještě článek popisující další části našeho algoritmu
do větších detailů společně s evaluací jeho celkového výkonu. A také ještě
druhý článek, který se zabývá biochemickými tématy jakými jsou nastavení
vstupních parametrů algoritmu, interpretace výsledků nebo vyhodnocení výsledků
testování naší metody v praxi na mnoha dvojicích proteinů a ligandů.

\section{Obecný přehled}
V této sekci popíšeme základní koncept naší metody. Detailní popis zmíněných
dvou částí, které jsou stěžejními tématy této práce, bude následovat v dalších
dvou kapitolách. Naše metoda je založena na řízeném iterativním pohybu ligandu
skrze tunel, díky čemuž se vyhneme časově náročnému výpočtu stochastického
pohybu ligandu, který je typický pro simulace molekulární dynamiky.

Prvním krokem naší metody je diskretizace tunelu jejímž výstupem je posloupnost
omezujících podmínek, díky kterým můžeme definovat pozici ligandu v tunelu a
tím pádem i směr jeho pohyb tunelem - dopředu a dozadu. Na takto diskretizovaném
tunelu je poté ligand iterativně dokován na sérii po sobě jdoucích pozic v tunelu,
což nám umožňuje simulovat proces navázání ligandu nebo naopak jeho uvolnění.

\subsection{Diskretizace tunelu}
K tomu abychom mohli s ligandem tunelem iterativně procházet, potřebujeme nějakým
způsobem omezit prostor, ve kterém může být ligand v každé iteraci umístěn.
Omezení se v našem případě realizuje pomocí posloupnosti koulí, které aproximují
geometrii tunelu. Takovouto aproximaci můžeme získat například pomocí nástroje
Caver \todo{reference}. Na vstupu algoritmu tedy máme zmíněnou posloupnost
koulí, kterou náš algoritmus transformuje na posloupnost $ n $ řezů
$ \theta_1, \dots, \theta_n $.
