\chapter*{Úvod}
\addcontentsline{toc}{chapter}{Úvod}

Porozumění interakcím biomolekul je
zásadní pro pochopení řady fundamentálních biochemických procesů, které
nalézají uplatnění v mnoha praktických odvětvích, jako je například farmacie.
\textit{Ligand} (malá molekula, např. léčivo) interaguje s \textit{proteinem}
v jeho \textit{aktivním místě}, kde dochází ke změně biochemických vlastností proteinu
či přeměně ligandu. Může tak dojít k inhibici funkce proteinu (který např.
jinak umožňuje viru napadnout buňku), nebo enzymatické přeměně substrátu na produkt
(např. rozklad toxické látky na neškodné produkty).
Simulace navázání ligandu (ligand vstoupí
do aktivního místa a vznikne stabilní komplex) a uvolnění ligandu či produktu ze stabilního komplexu
je užitečná v mnoha praktických
aplikacích, neboť nám poskytne aproximaci energie nutné pro transport ligandu či produktu z či do aktivního místa.
To v důsledku umožní například identifikovat ligandy, které se budou navazovat
v daném proteinu nejrychleji (při hledání léčiva), nebo modifikovat samotný protein tak,
aby k navázání docházelo snáze nebo hůře podle toho, jaké chování je žádoucí (optimalizace
proteinu pro rychlý rozklad toxických látek). Aktivní
místo proteinu je mnohdy ukryto hluboko uvnitř molekuly, takže aby se ligand
mohl navázat, musí nejprve projít tzv. \textit{tunelem}, který vede z povrchu proteinu
k jeho aktivnímu místu. V takovém případě je potřeba analyzovat, zda je
pravděpodobné, aby se ligand do aktivního místa skrze tunel dostal.

Chemické systémy, jakým je protein interagující s ligandem, se řídí druhým
termodynamickým zákonem a mají proto tendenci minimalizovat svou
celkovou potenciální energii. Prakticky to znamená, že nejpravděpodobnější
\textit{konformace} molekul (tj. pozice jejich atomů) jsou ty, které mají minimální
potenciální energii. Molekulární systém se ale může nacházet také v nějakém
lokálním energetickém minimu a s určitou pravděpodobností mezi těmito lokálními
minimy může přecházet. Pravděpodobnost přechodu pak ovlivňuje teplota systému a
velikost energetické bariéry, která musí být při přechodu překonána, přičemž platí
že čím menší bariéra, tím vyšší pravděpodobnost přechodu. Pokud známe
energetický profil průchodu ligandu tunelem proteinu, můžeme kvantifikovat
pravděpodobnost, s jakou ligand tunelem projde. Funkci, která pro zadanou konformaci
spočítá aproximaci její potenciální energii, nazýváme \textit{silové pole}. Analýzou
potenciální energie, kterou získáme ze silového pole, pak můžeme vypočítat
s jakou pravděpodobností se bude daná konformace vyskytovat v reálném chemickém
systému.
%Díky tomu je možné porovnat dva chemické komplexy tvořené
%dvojicemi protein-ligand (jakými jsou třeba dvě možné konformace ligandu
%v aktivním místě jednoho proteinu) a rozhodnout, který komplex má vyšší
%pravděpodobnost zformování v reálném systému.

Pokud chceme studovat navázání a uvolnění ligandu z aktivního místa, potřebujeme
být schopni vyhodnotit potenciální energii ligandu procházejícího tunelem
z povrchu proteinu do jeho aktivního místa a naopak. Ligand se v aktivním
místě proteinu naváže, pokud v tomto místě existuje silné energetické minimum
a v průběhu průchodu tunelem nenarazí na žádnou významnou energetickou bariéru
(gradient tunelu by měl být s menšími lokálními výkyvy stále klesající od
vstupu do tunelu až na jeho konec v aktivním místě). Pokud tunel obsahuje nějakou
silnou repulsivní bariéru, je pravděpodobné, že se ji ligandu nepodaří překonat
a tunelem neprojde. Poznamenejme, že energetický profil tunelu je pro každý
ligand unikátní a je tedy potřeba jej pro každou variaci ligandu vyhodnotit
znovu.

Pozice ligandu v aktivním místě proteinu se obvykle počítá pomocí takzvaného
molekulárního dokingu. Algoritmus molekulárního dokingu prochází konformační
prostor molekulárního komplexu protein-ligand a snaží se lokalizovat energetická
minima. Výstupem molekulárního dokingu je jedna nebo více konformací komplexu protein-ligand
společně s jejich potenciální energií. Díky tomu
uživatel získá informaci o tom, který ligand se na protein váže s nižší potenciální
energií a může se na základě této informace pokoušet ligand nebo protein
dále upravovat. Problémem je, že algoritmus molekulárního dokingu vypočítá pouze
statické pozice ligandu a proteinu – nevygeneruje \textit{trajektorii} ligandu
(pohyb ligandu tunelem v čase), což znamená, že tento algoritmus sám o sobě
nestačí na to, abychom mohli studovat přenos ligandu z povrchu proteinu
do jeho aktivního místa skrze jeho tunel, či naopak.

V této práci popíšeme některé části nové metody pro výpočet potenciální energie
trajektorie ligandu, která umožňuje studovat proces navázání i uvolnění ligandu
včetně jeho cesty skrze tunel k aktivnímu místu. Naše metoda je založena na
algoritmu molekulárního dokingu, který používá tak, že iterativně dokuje ligand
podél trajektorie tunelu a vyhodnocuje jeho potenciální energii. Náš doking
pracuje s hybridními silovými poli - jedná se o kombinaci chemického silového
pole, které se používá k výpočtu potenciální energie komplexu protein-ligand,
a omezujícího silového pole, které zmenšuje velikost prostoru potenciálních
konformací ligandu. Díky tomu je pozice ligandu v každém místě tunelu
omezena pouze na oblast definovaného prostoru. Skrze tunel zadaného proteinu
může existovat velmi mnoho cest, po kterých můžeme ligand provést. Prostor
všech potenciálních cest proto prohledáváme pomocí heuristického algoritmu
s backtrackingem. Naše metoda byla implementována v uživatelsky přívětivém
nástroji CaverDock, který je navržen tak, aby byl schopen maximálně využít
paralelních výpočetních architektur.

Tato práce se bude primárně soustředit na informaticko-matematickou stranu
daného problému: poskytneme obecný náhled na celý algoritmus a pak se budeme
soustředit zejména na dvě jeho části, za jejichž návrh a implementaci byl
zodpovědný autor této diplomové práce. Konkrétně se jedná o algoritmus pro
diskretizaci tunelu a algoritmus pro výpočet konvergence trajektorií ligandu.

Paralelně s touto prací vzniká ještě článek popisující další části našeho algoritmu
do větších detailů společně s evaluací jeho celkového výkonu. Rovněž je v přípravě
publikace, která se zabývá biochemickými tématy, jakými jsou nastavení
vstupních parametrů algoritmu, interpretace výsledků nebo vyhodnocení výsledků
testování naší metody v praxi na mnoha dvojicích proteinů a ligandů.

